\documentclass[10pt]{article}
\usepackage{geometry} % Pour passer au format A4
\geometry{hmargin=1cm, vmargin=1cm} % 

% Page et encodage
\usepackage[T1]{fontenc} % Use 8-bit encoding that has 256 glyphs
\usepackage[english,french]{babel} % Français et anglais
\usepackage[utf8]{inputenc} 

\usepackage{lmodern}
\setlength\parindent{0pt}

% Graphiques
\usepackage{graphicx,float,grffile}
\usepackage{pst-eucl, pst-plot, pgf,tikz} 
\usetikzlibrary{arrows}

% Maths et divers
\usepackage{amsmath,amsfonts,amssymb,amsthm,verbatim}
\usepackage{multicol,enumitem,url,eurosym,gensymb}

% Sections
\usepackage{sectsty} % Allows customizing section commands
\allsectionsfont{\centering \normalfont\scshape}

% Tête et pied de page

\usepackage{fancyhdr} 
\pagestyle{fancyplain} 

\fancyhead{} % No page header
\fancyfoot{}

\renewcommand{\headrulewidth}{0pt} % Remove header underlines
\renewcommand{\footrulewidth}{0pt} % Remove footer underlines

\newcommand{\horrule}[1]{\rule{\linewidth}{#1}} % Create horizontal rule command with 1 argument of height

%----------------------------------------------------------------------------------------
%	Début du document
%----------------------------------------------------------------------------------------

\begin{document}

%----------------------------------------------------------------------------------------
% RE-DEFINITION
%----------------------------------------------------------------------------------------
% MATHS
%-----------

\newtheorem{Definition}{Définition}
\newtheorem{Theorem}{Théorème}
\newtheorem{Proposition}{Propriété}

% MATHS
%-----------
\renewcommand{\labelitemi}{$\bullet$}
\renewcommand{\labelitemii}{$\circ$}
%----------------------------------------------------------------------------------------
%	Titre
%----------------------------------------------------------------------------------------

\setlength{\columnseprule}{0.5pt}

\section*{ie - Droites}
\begin{center}
  \textit{Théophraste- La plus coûteuse des dépenses, c’est la perte de temps.}
\end{center}
\horrule{2px}

\begin{multicols}{2}
  \subsection*{ex1}

  Soient les quatre fonctions : 
  $f_1 \mapsto 2x +1, 
  f_2 \mapsto 16, 
  f_3 \mapsto -2,3x \text{ et } 
  f_4 \mapsto 4x^2 +10$

  \begin{itemize}
  \item[1.] Quelle est la nature de ces quatre fonctions ?
  \item[2.] Pour chaque fonction, calculer l'image de 0.
  \item[3.] Calculer : $f_1(-2), f_2(-2), f_3(-2)  \text{ et } f_4(-2)$
  \end{itemize}
\end{multicols}
\horrule{0.5px}
\begin{multicols}{2}
  \subsection*{ex2}

  \definecolor{cqcqcq}{rgb}{0.75,0.75,0.75}
  \begin{tikzpicture}[line cap=round,line join=round,>=triangle 45,x=1.0cm,y=1.0cm,scale=0.6]
    \draw [color=cqcqcq,dash pattern=on 1pt off 1pt, xstep=1.0cm,ystep=1.0cm] (-5.66,-3.66) grid (10.39,5.14);
    \draw[->,color=black] (-5.66,0) -- (10.39,0);
    \foreach \x in {-5,-4,-3,-2,-1,1,2,3,4,5,6,7,8,9,10}
    \draw[shift={(\x,0)},color=black] (0pt,2pt) -- (0pt,-2pt) node[below] {\footnotesize $\x$};
    \draw[->,color=black] (0,-3.66) -- (0,5.14);
    \foreach \y in {-3,-2,-1,1,2,3,4,5}
    \draw[shift={(0,\y)},color=black] (2pt,0pt) -- (-2pt,0pt) node[left] {\footnotesize $\y$};
    \draw[color=black] (0pt,-10pt) node[right] {\footnotesize $0$};
    \clip(-5.66,-3.66) rectangle (10.39,5.14);
    \draw [line width=2pt,domain=-5.66:10.39] plot(\x,{(-8-4*\x)/-8});
    \draw [line width=2pt,domain=-5.66:10.39] plot(\x,{(-0-3*\x)/2});
    \begin{scriptsize}
      \draw[color=black] (-5,-2.5) node {\Large{$f_1$}};
      \draw[color=black] (-2,4.5) node {\Large{$f_2$}};
    \end{scriptsize}
  \end{tikzpicture}

  \begin{itemize}
  \item[1.] Déterminer l'image par la fonction $f_1$ de : 0 ; -2 ; 1 ; 6.
  \item[2.] Déterminer l’antécédent par la fonction $f_1$ de : 0 ; 2 ; -1 ; 3,5.
  \item[3.] Déterminer l'expression de $f_1$ sous la forme $ax + b$. 
  \item[4.] Déterminer l'image par la fonction $f_2$ de : 0 ; -2 ; 1 ; 2.
  \item[5.] Déterminer l’antécédent par la fonction $f_2$ de : 0 ; -1 ; 1 ; 3.
  \item[6.] Déterminer l'expression de $f_2$ sous la forme $ax + b$. 
  \end{itemize}

\end{multicols}

\subsection*{ex3}

\begin{itemize}
\item[1.] Déterminer l'expression de la droite passant par les points $A(1,2)$ et $B(3, 8)$ sous la forme  $f \mapsto ax +b$.
\item[2.] $g$ est une fonction affine telle que : g(-2) = 2 et g(6) = -4. Déterminer la fonction $g$.
\end{itemize}

\subsection*{ex4}

Tracer les droites dans le repère : 
$f_1 \mapsto 2x +1, 
f_2 \mapsto 3, 
f_3 \mapsto -2,5x \text{ et } 
f_4 \mapsto x -4$

\definecolor{cqcqcq}{rgb}{0.75,0.75,0.75}
\begin{tikzpicture}[line cap=round,line join=round,>=triangle 45,x=1.0cm,y=1.0cm,scale=0.8]
  \draw [color=cqcqcq,dash pattern=on 2pt off 2pt, xstep=0.5cm,ystep=0.5cm] (-8.44,-5.57) grid (15.32,5.84);
  \draw[->,color=black] (-8.44,0) -- (15.32,0);
  \foreach \x in {-8,-7,-6,-5,-4,-3,-2,-1,1,2,3,4,5,6,7,8,9,10,11,12,13,14,15}
  \draw[shift={(\x,0)},color=black] (0pt,2pt) -- (0pt,-2pt) node[below] {\footnotesize $\x$};
  \draw[->,color=black] (0,-5.57) -- (0,5.84);
  \foreach \y in {-5,-4,-3,-2,-1,1,2,3,4,5}
  \draw[shift={(0,\y)},color=black] (2pt,0pt) -- (-2pt,0pt) node[left] {\footnotesize $\y$};
  \draw[color=black] (0pt,-10pt) node[right] {\footnotesize $0$};
  \clip(-8.44,-5.57) rectangle (15.32,5.84);
\end{tikzpicture}

\newpage

\section*{ie - Droites}
\begin{center}
  \textit{Théophraste- La plus coûteuse des dépenses, c’est la perte de temps.}
\end{center}
\horrule{2px}

\begin{multicols}{2}
  \subsection*{ex1}

  Soient les quatre fonctions : 
  $f_1 \mapsto 4x + 2, 
  f_2 \mapsto 12x^2, 
  f_3 \mapsto -20 \text{ et } 
  f_4 \mapsto -5x$

  \begin{itemize}
  \item[1.] Quelle est la nature de ces quatre fonctions ?
  \item[2.] Pour chaque fonction, calculer l'image de 0.
  \item[3.] Calculer : $f_1(-2), f_2(-2), f_3(-2)  \text{ et } f_4(-2)$
  \end{itemize}
\end{multicols}
\horrule{0.5px}
\begin{multicols}{2}
  \subsection*{ex2}

  \definecolor{cqcqcq}{rgb}{0.75,0.75,0.75}
  \begin{tikzpicture}[line cap=round,line join=round,>=triangle 45,x=1.0cm,y=1.0cm,scale=0.6]
    \draw [color=cqcqcq,dash pattern=on 1pt off 1pt, xstep=1.0cm,ystep=1.0cm] (-5.66,-3.66) grid (10.39,5.14);
    \draw[->,color=black] (-5.66,0) -- (10.39,0);
    \foreach \x in {-5,-4,-3,-2,-1,1,2,3,4,5,6,7,8,9,10}
    \draw[shift={(\x,0)},color=black] (0pt,2pt) -- (0pt,-2pt) node[below] {\footnotesize $\x$};
    \draw[->,color=black] (0,-3.66) -- (0,5.14);
    \foreach \y in {-3,-2,-1,1,2,3,4,5}
    \draw[shift={(0,\y)},color=black] (2pt,0pt) -- (-2pt,0pt) node[left] {\footnotesize $\y$};
    \draw[color=black] (0pt,-10pt) node[right] {\footnotesize $0$};
    \clip(-5.66,-3.66) rectangle (10.39,5.14);
    \draw [line width=2pt,domain=-5.66:10.39] plot(\x,{(-8-4*\x)/-8});
    \draw [line width=2pt,domain=-5.66:10.39] plot(\x,{(-0-3*\x)/2});
    \begin{scriptsize}
      \draw[color=black] (-5,-2.5) node {\Large{$f_1$}};
      \draw[color=black] (-2,4.5) node {\Large{$f_2$}};
    \end{scriptsize}
  \end{tikzpicture}

  \begin{itemize}
  \item[1.] Déterminer l'image par la fonction $f_1$ de : 0 ; -2 ; 3 ; 8.
  \item[2.] Déterminer l’antécédent par la fonction $f_1$ de : 0 ; 3 ; -1 ; 2,5.
  \item[3.] Déterminer l'expression de $f_1$ sous la forme $ax + b$. 
  \item[4.] Déterminer l'image par la fonction $f_2$ de : 0 ; -2 ; 1 ; 2.
  \item[5.] Déterminer l’antécédent par la fonction $f_2$ de : 0 ; -2 ; 2 ; 3.
  \item[6.] Déterminer l'expression de $f_2$ sous la forme $ax + b$. 
  \end{itemize}

\end{multicols}

\subsection*{ex3}

\begin{itemize}
\item[1.] Déterminer l'expression de la droite passant par les points $A(2,3)$ et $B(5, 10)$ sous la forme  $f \mapsto ax +b$.
\item[2.] $g$ est une fonction affine telle que : g(-4) = -6 et g(5) = 3. Déterminer la fonction $g$.
\end{itemize}

\subsection*{ex4}

Tracer les droites dans le repère : 
$f_1 \mapsto 2x + 2, 
f_2 \mapsto 5, 
f_3 \mapsto -1,5x \text{ et } 
f_4 \mapsto x - 3$

\definecolor{cqcqcq}{rgb}{0.75,0.75,0.75}
\begin{tikzpicture}[line cap=round,line join=round,>=triangle 45,x=1.0cm,y=1.0cm,scale=0.8]
  \draw [color=cqcqcq,dash pattern=on 2pt off 2pt, xstep=0.5cm,ystep=0.5cm] (-8.44,-5.57) grid (15.32,5.84);
  \draw[->,color=black] (-8.44,0) -- (15.32,0);
  \foreach \x in {-8,-7,-6,-5,-4,-3,-2,-1,1,2,3,4,5,6,7,8,9,10,11,12,13,14,15}
  \draw[shift={(\x,0)},color=black] (0pt,2pt) -- (0pt,-2pt) node[below] {\footnotesize $\x$};
  \draw[->,color=black] (0,-5.57) -- (0,5.84);
  \foreach \y in {-5,-4,-3,-2,-1,1,2,3,4,5}
  \draw[shift={(0,\y)},color=black] (2pt,0pt) -- (-2pt,0pt) node[left] {\footnotesize $\y$};
  \draw[color=black] (0pt,-10pt) node[right] {\footnotesize $0$};
  \clip(-8.44,-5.57) rectangle (15.32,5.84);
\end{tikzpicture}

\end{document}
