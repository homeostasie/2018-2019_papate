\documentclass[12pt]{article}
\usepackage{geometry} % Pour passer au format A4
\geometry{hmargin=1cm, vmargin=1cm} % 

% Page et encodage
\usepackage[T1]{fontenc} % Use 8-bit encoding that has 256 glyphs
\usepackage[english,french]{babel} % Français et anglais
\usepackage[utf8]{inputenc} 

\usepackage{lmodern}
\setlength\parindent{0pt}

% Graphiques
\usepackage{graphicx,float,grffile}

% Maths et divers
\usepackage{amsmath,amsfonts,amssymb,amsthm,verbatim}
\usepackage{multicol,enumitem,url,eurosym,gensymb}

% Sections
\usepackage{sectsty} % Allows customizing section commands
\allsectionsfont{\centering \normalfont\scshape}

% Tête et pied de page

\usepackage{fancyhdr} 
\pagestyle{fancyplain} 

\fancyhead{} % No page header
\fancyfoot{}

\renewcommand{\headrulewidth}{0pt} % Remove header underlines
\renewcommand{\footrulewidth}{0pt} % Remove footer underlines

\newcommand{\horrule}[1]{\rule{\linewidth}{#1}} % Create horizontal rule command with 1 argument of height

%----------------------------------------------------------------------------------------
%   Début du document
%----------------------------------------------------------------------------------------

\begin{document}

%----------------------------------------------------------------------------------------
% RE-DEFINITION
%----------------------------------------------------------------------------------------
% MATHS
%-----------

\newtheorem{Definition}{Définition}
\newtheorem{Theorem}{Théorème}
\newtheorem{Proposition}{Propriété}

% MATHS
%-----------
\renewcommand{\labelitemi}{$\bullet$}
\renewcommand{\labelitemii}{$\circ$}
%----------------------------------------------------------------------------------------
%   Titre
%----------------------------------------------------------------------------------------

\setlength{\columnseprule}{1pt}

\horrule{2px}
\section*{Chapitre 4 - Petits problèmes de mathématiques}
\horrule{2px}

\begin{enumerate}
\item[1.] Connaissances
  \begin{itemize}
  \item Opération Puissance positive : $x^n$.
  \item Opération Puissance négative : $x^{-n}$.
  \item Connaître les simplifications avec des puissances.
  \item Connaître l'écriture scientifique.
  \end{itemize}
\item[2.] Compétences
  \begin{itemize}
  \item Calculer avec des puissances.
  \item Comprendre quelques limites de sa calculatrice.
  \end{itemize}
\end{enumerate}

\section*{1 - Vitesse moyenne}

Une vitesse est une grandeur composée d'une distance et d'un temps.

\subsection*{Distance}

$$1 km = 1\,000m$$

On en déduit : 
	\begin{eqnarray*}
  2km &=& 2 \,000m \\
  25km &=& 25 \, 000m \\
  0.5 km &=& 500m
\end{eqnarray*}

$$1m = 1\div 1000 = \frac{1}{1000} m$$

On en déduit : 
	\begin{eqnarray*}
  2m &=& 2 \div 1\,000 km \\
  17m &=& 17 \div 1\, 000 km \\
  0.4 m &=& 0.4 \div 1\, 000 km
\end{eqnarray*}

\subsection*{Temps}

$$ 1h = 60min $$

On en déduit : 
	\begin{eqnarray*}
  2h &=& 2 \times 60 = 120 min \\
  0.5h &=& 0.5 \times 60min = 30 min \\
  0.3h &=& 0.3 \times 60min = 18 min \\
  5.3h &=& 5.3 \times 60min = 318 min \\
  5h33min &=& 5 \times 60min + 33min = 300 + 33 min = 333min\\
\end{eqnarray*}

Il en est de même pour les secondes.

\begin{eqnarray*}
  1min = 60s \\
  1h = 60 \times 60s = 3600s
\end{eqnarray*}

\section*{2 - Pourcentage}

\section*{3 - Echelle}

\end{document}
