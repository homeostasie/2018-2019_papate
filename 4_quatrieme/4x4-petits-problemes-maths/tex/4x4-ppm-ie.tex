\documentclass[12pt]{article}
\usepackage{geometry} % Pour passer au format A4
\geometry{hmargin=1cm, vmargin=1cm} % 

% Page et encodage
\usepackage[T1]{fontenc} % Use 8-bit encoding that has 256 glyphs
\usepackage[english,french]{babel} % Français et anglais
\usepackage[utf8]{inputenc} 

\usepackage{lmodern}
\setlength\parindent{0pt}

% Graphiques
\usepackage{graphicx,float,grffile}
\usepackage{tikz}

% Maths et divers
\usepackage{amsmath,amsfonts,amssymb,amsthm,verbatim}
\usepackage{multicol,enumitem,url,eurosym,gensymb}
\usepackage{multido}

% Sections
\usepackage{sectsty} % Allows customizing section commands
\allsectionsfont{\centering \normalfont\scshape}

% Tête et pied de page
\usepackage{fancyhdr} 
\pagestyle{fancyplain} 
\fancyhead{} % No page header
\fancyfoot{}

\renewcommand{\headrulewidth}{0pt} % Remove header underlines
\renewcommand{\footrulewidth}{0pt} % Remove footer underlines

\newcommand{\horrule}[1]{\rule{\linewidth}{#1}} % Create horizontal rule command with 1 argument of height

\newcommand{\Pointilles}[1]{%
  \par\nobreak
  \noindent\rule{0pt}{1.5\baselineskip}% Provides a larger gap between the preceding paragraph and the dots
  \multido{}{#1}{\noindent\makebox[\linewidth]{\dotfill}\endgraf}% ... dotted lines ...
  \bigskip% Gap between dots and next paragraph
}
%----------------------------------------------------------------------------------------
%	Début du document
%----------------------------------------------------------------------------------------

\begin{document}

%----------------------------------------------------------------------------------------
% RE-DEFINITION
%----------------------------------------------------------------------------------------
% MATHS
%-----------

\newtheorem{Definition}{Définition}
\newtheorem{Theorem}{Théorème}
\newtheorem{Proposition}{Propriété}

% MATHS
%-----------
\renewcommand{\labelitemi}{$\bullet$}
\renewcommand{\labelitemii}{$\circ$}
%----------------------------------------------------------------------------------------
%	Titre
%----------------------------------------------------------------------------------------

\setlength{\columnseprule}{1pt}

\textbf{Nom, Prénom : }

\section*{ie 5 - Petits Problèmes de Maths}
\begin{center}
  \textit{René Char - L’essentiel est sans cesse menacé par l’insignifiant.}
\end{center}
\horrule{2px}


\begin{multicols}{2}
\subsection*{Ex1}
Un glacier avance de 17 m par jour. De combien aura-t-il avancé :
\begin{itemize}
\item[1a.] au bout d’un mois de 30 jours ?
\item[1b.] au bout d’une année de 365 jours ?
\item[1c.] au bout de deux siècles ?
\end{itemize}

\subsection*{Ex2}
Un automobiliste roule à allure constante. Il parcourt 140 km en une heure. Quelle distance parcourt-il en :
\begin{itemize}
\item[2a.] 2 h ? 	
\item[2b.] 3 h 30 min ? 	
\item[2c.] c. 33 min ? 
\end{itemize}
\end{multicols}

\begin{multicols}{2}
\subsection*{Ex3}
\begin{itemize}
\item[3a.] L'ouragan Lothar touche le Finistère le 26 décembre à 2 h du matin et atteint Strasbourg (soit 900 km plus loin) vers 13 h. \\
Calcule la vitesse moyenne à laquelle cette tempête a traversé la France.
\item[3b.] L'ouragan Martin aborde le sud Finistère le 27 décembre vers 15 h et se propage à 75 km/h sur une distance égale à celle de Lothar. \\
À quelle heure arrive-t-il en Alsace ?
\end{itemize}

\subsection*{Ex4}
Une entreprise a produit 250 tonnes de clous.\\
Elle a vendu : 
\begin{itemize}
\item Un quart de sa production sur le marché national
\item 25 \% sur le marché européen
\item 12 \% sur le marché américain
\item Le reste sur le marché asiatique.
\end{itemize}

 Dans chaque cas, calcule la production en tonnes correspondante.
\end{multicols}

\begin{multicols}{2}
\subsection*{Ex5}
 Au collège de Noémie, le foyer socio-éducatif (FSE) prend en charge 25 \% du financement des voyages scolaires alors que dans celui de Didier, pour un voyage de 180 \euro, le FSE a donné 54 \euro.

\begin{itemize}
\item[5a.] Si Noémie participe à un voyage qui coûte 230 \euro, quel montant est-il pris en charge par son FSE ?
\item[5b.] En proportion, dans quel collège le FSE participe-t-il le plus au financement des voyages ?
\end{itemize}

\subsection*{Ex6}
Recopier et completer les phrases suivantes.
\begin{itemize}
\item[6a.] 1 cm sur le plan correspond à 40 cm en réalité.\\
L’échelle du plan est donc : 
\item[6b.] 1 cm sur le plan correspond à 80 000 cm en réalité. \\
L’échelle du plan est donc : 
\item[6c.] 1 cm sur le plan correspond à 7 km en réalité.\\
L’échelle du plan est donc : 
\end{itemize}
\end{multicols}
\begin{multicols}{2}
\subsection*{Ex7}
\begin{itemize}
\item[7a.] Sur un plan de maison à l’échelle 1/125, la salle à manger est représentée par un rectangle de 7.2 cm de long sur 6.3 cm de large. \\
Quelles sont les dimensions réelles de cette pièce en mètre?
\item[7b.] La chambre mesure en réalité 7m de long par 4m de large. \\
Quelles sont les dimensions sur le plan de cette pièce en cm?
\end{itemize}

\subsection*{Ex8}
Sur le plan d'une maison, les portes sont représentées par un segment de 1,4 cm de long. \\
En réalité, elles sont larges de 0,60 m. Quelle est l'échelle de ce plan ?
\end{multicols}


\newpage

\textbf{Nom, Prénom : }

\section*{ie 5 - Petits Problèmes de Maths}
\begin{center}
  \textit{René Char - L’essentiel est sans cesse menacé par l’insignifiant.}
\end{center}
\horrule{2px}


\begin{multicols}{2}
\subsection*{Ex1}
Un glacier avance de 27 m par jour. De combien aura-t-il avancé :
\begin{itemize}
\item[1a.] au bout d’un mois de 30 jours ?
\item[1b.] au bout d’une année de 365 jours ?
\item[1c.] au bout de trois siècles ?
\end{itemize}

\subsection*{Ex2}
Un automobiliste roule à allure constante. Il parcourt 120 km en une heure. Quelle distance parcourt-il en :
\begin{itemize}
\item[2a.] 4 h ? 	
\item[2b.] 2 h 30 min ? 	
\item[2c.] c. 27 min ? 
\end{itemize}
\end{multicols}

\begin{multicols}{2}
\subsection*{Ex3}
\begin{itemize}
\item[3a.] L'ouragan Lothar touche le Finistère le 25 novembre à 3 h du matin et atteint Strasbourg (soit 800 km plus loin) vers 12 h. \\
Calcule la vitesse moyenne à laquelle cette tempête a traversé la France.
\item[3b.] L'ouragan Martin aborde le sud Finistère le 24 décembre vers 14 h et se propage à 85 km/h sur une distance égale à celle de Lothar. \\
À quelle heure arrive-t-il en Alsace ?
\end{itemize}

\subsection*{Ex4}
Une entreprise a produit 350 tonnes de clous.\\
Elle a vendu : 
\begin{itemize}
\item Un quart de sa production sur le marché national
\item 35 \% sur le marché européen
\item 22 \% sur le marché américain
\item Le reste sur le marché asiatique.
\end{itemize}

 Dans chaque cas, calcule la production en tonnes correspondante.
\end{multicols}

\begin{multicols}{2}
\subsection*{Ex5}
 Au collège de Noémie, le foyer socio-éducatif (FSE) prend en charge 15 \% du financement des voyages scolaires alors que dans celui de Didier, pour un voyage de 280 \euro, le FSE a donné 84 \euro.

\begin{itemize}
\item[5a.] Si Noémie participe à un voyage qui coûte 330 \euro, quel montant est-il pris en charge par son FSE ?
\item[5b.] En proportion, dans quel collège le FSE participe-t-il le plus au financement des voyages ?
\end{itemize}

\subsection*{Ex6}
Recopier et completer les phrases suivantes.
\begin{itemize}
\item[6a.] 1 cm sur le plan correspond à 50 cm en réalité.\\
L’échelle du plan est donc : 
\item[6b.] 1 cm sur le plan correspond à 5 000 cm en réalité. \\
L’échelle du plan est donc : 
\item[6c.] 1 cm sur le plan correspond à 2 km en réalité.\\
L’échelle du plan est donc : 
\end{itemize}
\end{multicols}
\begin{multicols}{2}
\subsection*{Ex7}
\begin{itemize}
\item[7a.] Sur un plan de maison à l’échelle 1/115, la salle à manger est représentée par un rectangle de 7.4 cm de long sur 6.7 cm de large. \\
Quelles sont les dimensions réelles de cette pièce en mètre?
\item[7b.] La chambre mesure en réalité 6m de long par 5m de large. \\
Quelles sont les dimensions sur le plan de cette pièce en cm?
\end{itemize}

\subsection*{Ex8}
Sur le plan d'une maison, les portes sont représentées par un segment de 1,2 cm de long. \\
En réalité, elles sont larges de 0,80 m. Quelle est l'échelle de ce plan ?
\end{multicols}
\end{document}
