\documentclass[11pt]{article}
\usepackage{geometry} % Pour passer au format A4
\geometry{hmargin=1cm, vmargin=1cm} % 

% Page et encodage
\usepackage[T1]{fontenc} % Use 8-bit encoding that has 256 glyphs
\usepackage[english,francais]{babel} % Français et anglais
\usepackage[utf8]{inputenc} 

\usepackage{lmodern}
\setlength\parindent{0pt}

% Graphiques
\usepackage{graphicx, float}


% Maths et divers
\usepackage{amsmath,amsfonts,amssymb,amsthm,verbatim}
\usepackage{multicol,enumitem,url,eurosym}

% Sections
\usepackage{sectsty} % Allows customizing section commands
\allsectionsfont{\centering \normalfont\scshape}

% Tête et pied de page

\usepackage{fancyhdr} 
\pagestyle{fancyplain} 

\fancyhead{} % No page header
\fancyfoot{}

\renewcommand{\headrulewidth}{0pt} % Remove header underlines
\renewcommand{\footrulewidth}{0pt} % Remove footer underlines

\newcommand{\horrule}[1]{\rule{\linewidth}{#1}} % Create horizontal rule command with 1 argument of height

%----------------------------------------------------------------------------------------
%	Début du document
%----------------------------------------------------------------------------------------

\begin{document}

%----------------------------------------------------------------------------------------
% RE-DEFINITION
%----------------------------------------------------------------------------------------
% MATHS
%-----------

\newtheorem{Definition}{Définition}
\newtheorem{Theorem}{Théorème}
\newtheorem{Proposition}{Propriété}

% MATHS
%-----------
\renewcommand{\labelitemi}{$\bullet$}
\renewcommand{\labelitemii}{$\circ$}
%----------------------------------------------------------------------------------------
%	Titre
%----------------------------------------------------------------------------------------

\setlength{\columnseprule}{1pt}

\horrule{2px}
\section*{Chapitre 1 - Théorème de Pythagore}
\horrule{2px}

\begin{enumerate}
    \item[1.] Connaissances
    \begin{itemize}
        \item Fonction Carré : $x^2$
        \item Fonction Racine carré : $\sqrt{x}$ 
        \item Connaitre le Théorème
    \end{itemize}
    \item[2.] Compétences
    \begin{itemize}
        \item Savoir quand utiliser le théorème de Pythagore.
        \item Savoir calculer une longueur à l'aide du théorème de Pythagore.
    \end{itemize}
\end{enumerate}

\section*{1 - Le carré}

\begin{Definition}
    Un carré possède 4 côtés de même longueur et 4 angle droit.
\end{Definition}

Soit un carré de côté $c$.
\begin{itemize}
    \item Périmètre : $4 \times c = 4c$
    \item Aire : $c \times c = c^2$ 
\end{itemize}

\section*{2 - Modéliser}

\textit{Quand utiliser le théorème de Pythagore ?}\\

On utilise le théorème de Pythagore quand on cherche à \textbf{calculer la longueur d'un côté} dans un \textbf{triangle rectangle} en connaissant déjà les \textbf{deux autres côtés}.



\section*{3 - Théorèmes}

\begin{Definition}
Dans un triangle rectangle, le plus grand côté, opposé à l'angle droit s'appelle \textbf{l'hypoténuse}.
\end{Definition}



\begin{Theorem}{Énoncé géométrique}\\
Dans la triangle rectangle, l'aire du carré issue de l'hypoténuse est égale à la somme des aires des carrés issues des deux autres côtés.
\end{Theorem}

\begin{Theorem}{Énoncé général}\\
Dans un triangle rectangle, le carré de l'hypténuse est égale à la somme des autres côtés au carré.
\end{Theorem}

\begin{Theorem}{Énoncé exercice}\\
    \begin{enumerate}
        \item[1.] Dans le triangle ABC rectangle en A.
        \item[2.] D'après le théorème de Pythagore.
        \item[3.] $BC^2 = AB^2 + AC^2$
    \end{enumerate}
    
    \begin{figure}[H]
        \centering
        \includegraphics[width=0.5\linewidth]{4x1_pythagore/sources/4x1_pythagore-thm-exo.pdf}
    \end{figure}
\end{Theorem}



\subsection*{Démonsrations}

\url{https://www.geogebra.org/m/PujhpZtM}

\url{https://www.youtube.com/watch?v=CAkMUdeB06o}

Euclide - Les éléments - Livre 1 - p48


\section*{4 - Calcul du troisième côté}

Il faut être vigilent (prudent) quand on utilise le théorème de Pythagore car il y a une différence selon si le côté cherché est l'hypoténuse ou les autres..

\subsection*{Recherche d'un hypoténuse}

\begin{figure}[H]
    \centering
    \includegraphics[width=0.5\linewidth]{4x1_pythagore/sources/4x1_pythagore-re-h.pdf}
\end{figure}

\subsection*{Recherche d'un petit côté}

\begin{figure}[H]
    \centering
    \includegraphics[width=0.5\linewidth]{4x1_pythagore/sources/4x1_pythagore-re-p.pdf}
\end{figure}

\end{document}
