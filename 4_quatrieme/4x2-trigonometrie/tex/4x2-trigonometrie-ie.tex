\documentclass[12pt]{article}
\usepackage{geometry} % Pour passer au format A4
\geometry{hmargin=1cm, vmargin=1cm} % 

% Page et encodage
\usepackage[T1]{fontenc} % Use 8-bit encoding that has 256 glyphs
\usepackage[english,french]{babel} % Français et anglais
\usepackage[utf8]{inputenc} 

\usepackage{lmodern}
\setlength\parindent{0pt}

% Graphiques
\usepackage{graphicx,float,grffile}

% Maths et divers
\usepackage{amsmath,amsfonts,amssymb,amsthm,verbatim}
\usepackage{multicol,enumitem,url,eurosym,gensymb}

% Sections
\usepackage{sectsty} % Allows customizing section commands
\allsectionsfont{\centering \normalfont\scshape}

% Tête et pied de page

\usepackage{fancyhdr} 
\pagestyle{fancyplain} 

\fancyhead{} % No page header
\fancyfoot{}

\renewcommand{\headrulewidth}{0pt} % Remove header underlines
\renewcommand{\footrulewidth}{0pt} % Remove footer underlines

\newcommand{\horrule}[1]{\rule{\linewidth}{#1}} % Create horizontal rule command with 1 argument of height

%----------------------------------------------------------------------------------------
%	Début du document
%----------------------------------------------------------------------------------------

\begin{document}

%----------------------------------------------------------------------------------------
% RE-DEFINITION
%----------------------------------------------------------------------------------------
% MATHS
%-----------

\newtheorem{Definition}{Définition}
\newtheorem{Theorem}{Théorème}
\newtheorem{Proposition}{Propriété}

% MATHS
%-----------
\renewcommand{\labelitemi}{$\bullet$}
\renewcommand{\labelitemii}{$\circ$}
%----------------------------------------------------------------------------------------
%	Titre
%----------------------------------------------------------------------------------------

\setlength{\columnseprule}{1pt}

\section*{ie 3 - Trigonométrie}
\begin{center}
  \textit{Vincent Van Gogh - La normalité est une route pavée : on y marche aisément mais les fleurs n’y poussent pas.}
\end{center}
\horrule{2px}

\subsection*{Ex1 - Restituer}

Restituer les côtés : hypoténuse (HYP), adjacent (ADJ) et opposé (OPP) dans chacun des triangles suivants.

\subsection*{Ex2 - Restituer}

En citant la condition d'utilisation dans un triangle des relations de trigonométrie, restituer les trois relations trigonométriques sinus (sin), cosinus (cos) et tangente (tan) en fonction des côtés Hypoténuse (HYP), Adjacent (ADJ) et Opposé (OPP). Ecrivez également la phrase \textit{mémoire} qui permet de s'en rappeler.

\subsection*{Ex3 - Restituer}

Dans les deux triangles suivants, écrire les trois relations trigonométriques sinus (sin), cosinus (cos) et tangente (tan) en fonction des côtés.

\subsection*{Ex4 - Modéliser}

Modéliser si on doit utiliser sinus (sin), cosinus (cos) ou tangente (tan). Justifier en écrivant la relation trigonométrique en fonction des côtés.

\subsection*{Ex5 - Calculer}

Calculer à l'aide de la calculatrice et remplir le tableau. Mettre une valeur approchée arrondie à deux chiffres après la virgule. 


\subsection*{Bonus}


  \begin{figure}[H]
    \centering
    \includegraphics[width=0.6 \linewidth]{4x1-pythagore/sources/cah.jpg}
  \end{figure}


\end{document}
