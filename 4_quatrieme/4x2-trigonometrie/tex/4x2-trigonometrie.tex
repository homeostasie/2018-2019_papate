\documentclass[12pt]{article}
\usepackage{geometry} % Pour passer au format A4
\geometry{hmargin=1cm, vmargin=1cm} % 

% Page et encodage
\usepackage[T1]{fontenc} % Use 8-bit encoding that has 256 glyphs
\usepackage[english,french]{babel} % Français et anglais
\usepackage[utf8]{inputenc} 

\usepackage{lmodern}
\setlength\parindent{0pt}

% Graphiques
\usepackage{graphicx,float,grffile}

% Maths et divers
\usepackage{amsmath,amsfonts,amssymb,amsthm,verbatim}
\usepackage{multicol,enumitem,url,eurosym}

% Sections
\usepackage{sectsty} % Allows customizing section commands
\allsectionsfont{\centering \normalfont\scshape}

% Tête et pied de page

\usepackage{fancyhdr} 
\pagestyle{fancyplain} 

\fancyhead{} % No page header
\fancyfoot{}

\renewcommand{\headrulewidth}{0pt} % Remove header underlines
\renewcommand{\footrulewidth}{0pt} % Remove footer underlines

\newcommand{\horrule}[1]{\rule{\linewidth}{#1}} % Create horizontal rule command with 1 argument of height

%----------------------------------------------------------------------------------------
%	Début du document
%----------------------------------------------------------------------------------------

\begin{document}

%----------------------------------------------------------------------------------------
% RE-DEFINITION
%----------------------------------------------------------------------------------------
% MATHS
%-----------

\newtheorem{Definition}{Définition}
\newtheorem{Theorem}{Théorème}
\newtheorem{Proposition}{Propriété}

% MATHS
%-----------
\renewcommand{\labelitemi}{$\bullet$}
\renewcommand{\labelitemii}{$\circ$}
%----------------------------------------------------------------------------------------
%	Titre
%----------------------------------------------------------------------------------------

\setlength{\columnseprule}{1pt}

\horrule{2px}
\section*{Chapitre 2 - Trigonometrie}
\horrule{2px}

\begin{enumerate}
	\item[1.] Connaissances
	      \begin{itemize}
		  	  \item Les côtés dans un triangle rectangle. 
		      \item Les relations trigonométriques Sinus, Cosinus, Tangente : SohCahToa
	      \end{itemize}
	\item[2.] Compétences
	      \begin{itemize}
		      \item Savoir quand utiliser la trigonométrie.
			  \item Savoir quelle relation trigonométrique utiliser.
		      \item Savoir calculer une longueur à l'aide de la trigonométrie.
	      \end{itemize}
\end{enumerate}

\section*{1 - Contexte}
 
Une des limites du théorème de Pythagore est qu'il demande deux mesures de longueurs. Or toutes les longueurs ne sont pas facilement mesurables. Exemple : la hauteur d'un immeuble. 

\end{document}
