\documentclass[12pt]{article}
\usepackage{geometry} % Pour passer au format A4
\geometry{hmargin=1cm, vmargin=1cm} % 

% Page et encodage
\usepackage[T1]{fontenc} % Use 8-bit encoding that has 256 glyphs
\usepackage[english,french]{babel} % Français et anglais
\usepackage[utf8]{inputenc} 

\usepackage{lmodern}
\setlength\parindent{0pt}

% Graphiques
\usepackage{graphicx,float,grffile}

% Maths et divers
\usepackage{amsmath,amsfonts,amssymb,amsthm,verbatim}
\usepackage{multicol,enumitem,url,eurosym}

% Sections
\usepackage{sectsty} % Allows customizing section commands
\allsectionsfont{\centering \normalfont\scshape}

% Tête et pied de page

\usepackage{fancyhdr} 
\pagestyle{fancyplain} 

\fancyhead{} % No page header
\fancyfoot{}

\renewcommand{\headrulewidth}{0pt} % Remove header underlines
\renewcommand{\footrulewidth}{0pt} % Remove footer underlines

\newcommand{\horrule}[1]{\rule{\linewidth}{#1}} % Create horizontal rule command with 1 argument of height

%----------------------------------------------------------------------------------------
%	Début du document
%----------------------------------------------------------------------------------------

\begin{document}

%----------------------------------------------------------------------------------------
% RE-DEFINITION
%----------------------------------------------------------------------------------------
% MATHS
%-----------

\newtheorem{Definition}{Définition}
\newtheorem{Theorem}{Théorème}
\newtheorem{Proposition}{Propriété}

% MATHS
%-----------
\renewcommand{\labelitemi}{$\bullet$}
\renewcommand{\labelitemii}{$\circ$}
%----------------------------------------------------------------------------------------
%	Titre
%----------------------------------------------------------------------------------------

\setlength{\columnseprule}{1pt}

\horrule{2px}
\section*{Chapitre 1 - Théorème de Pythagore}
\horrule{2px}

\subsection*{Programme des exercices}

\subsection*{Calculer des carrés}

\begin{itemize}
	\item \textit{On donne des côtés, on cherche les aires des carrés.}
	\item \textit{On donne des nombres entiers, on cherche les carrés correspondants.}
	\item \textit{On donne des nombres décimaux, on cherche les carrés correspondants.}
	\item \textit{Pour les plus rapides : fractions, nombres négatifs, ...}
\end{itemize}

\subsection*{Calculer des racines carrés}

\begin{itemize}
	\item \textit{On donne des aires de carré, on cherche les côtés.}
	\item \textit{On donne des carrés parfaits, on cherche les racines.}
	\item \textit{On donne des nombres décimaux, on cherche les racines.}
	\item \textit{Pour les plus rapides : fractions, nombres négatifs, ...}
\end{itemize}

\subsection*{Écrire le théorème}

\begin{itemize}
	\item \textit{On donne des triangles rectangles avec des valeurs, on écrit l'égalité et on l'a vérifie.}
	\item \textit{On donne des triangles rectangles avec des lettres, on écrit l'égalité.}
	\item \textit{On donne des situations de problèmes, on dit si oui ou non on peut utiliser le théorème de Pythagore.}
\end{itemize}

\subsection*{Calcul de longueurs manquantes}

\begin{itemize}
	\item \textit{Triangles rectangles, on donne deux côtés, on dit si on doit additionner ou soustraire.}
	\item \textit{Triangles rectangles, on donne deux côtés, on calcule le troisième côté.}
\end{itemize}

\subsection*{Problèmes}

\end{document}
