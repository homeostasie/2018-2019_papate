\documentclass[12pt]{article}
\usepackage{geometry} % Pour passer au format A4
\geometry{hmargin=1cm, vmargin=1cm} % 

% Page et encodage
\usepackage[T1]{fontenc} % Use 8-bit encoding that has 256 glyphs
\usepackage[english,french]{babel} % Français et anglais
\usepackage[utf8]{inputenc} 

\usepackage{lmodern}
\setlength\parindent{0pt}

% Graphiques
\usepackage{graphicx,float,grffile}
\usepackage{pst-eucl, pst-plot} 

% Maths et divers
\usepackage{amsmath,amsfonts,amssymb,amsthm,verbatim}
\usepackage{multicol,enumitem,url,eurosym,gensymb}

% Sections
\usepackage{sectsty} % Allows customizing section commands
\allsectionsfont{\centering \normalfont\scshape}

% Tête et pied de page

\usepackage{fancyhdr} 
\pagestyle{fancyplain} 

\fancyhead{} % No page header
\fancyfoot{}

\renewcommand{\headrulewidth}{0pt} % Remove header underlines
\renewcommand{\footrulewidth}{0pt} % Remove footer underlines

\newcommand{\horrule}[1]{\rule{\linewidth}{#1}} % Create horizontal rule command with 1 argument of height
\newcommand{\Pointille}[1][3]{\multido{}{#1}{ \makebox[\linewidth]{\dotfill}\\[\parskip]}}

%----------------------------------------------------------------------------------------
%	Début du document
%----------------------------------------------------------------------------------------

\begin{document}

%----------------------------------------------------------------------------------------
% RE-DEFINITION
%----------------------------------------------------------------------------------------
% MATHS
%-----------

\newtheorem{Definition}{Définition}
\newtheorem{Theorem}{Théorème}
\newtheorem{Proposition}{Propriété}

% MATHS
%-----------
\renewcommand{\labelitemi}{$\bullet$}
\renewcommand{\labelitemii}{$\circ$}
%----------------------------------------------------------------------------------------
%	Titre
%----------------------------------------------------------------------------------------

\setlength{\columnseprule}{1pt}

\section*{ie - Équations}
\begin{center}
  \textit{Théophraste- La plus coûteuse des dépenses, c’est la perte de temps.}
\end{center}
\textbf{Nom, Prenom :}\\
\horrule{2px}

\subsection*{ex1 - Calculs  à trou}
\textit{Trouver le nombre manquant.}

\begin{multicols}{3}\noindent
    \begin{enumerate}
    \item $-14 - \ldots\ldots = -8$
    \item $-8 + \left( -9\right) = \ldots\ldots$
    \item $6 \times \ldots\ldots = -48$
    \item $\ldots\ldots + 9 = 12$
    \item $-28 \div 7 = \ldots\ldots$
    \item $-32 \div \ldots\ldots = -4$
    \item $-10 \times 5 = \ldots\ldots$
    \item $2 - 3 = \ldots\ldots$
    \item $-21 \div 7 = \ldots\ldots$
    \item $-14 \div \left( -7\right) = \ldots\ldots$
    \item $-15 - \ldots\ldots = -7$
    \item $8 - \ldots\ldots = -1$
    \item $-8 \times \ldots\ldots = 40$
    \item $50 \div \left( -5\right) = \ldots\ldots$
    \item $10 \times 7 = \ldots\ldots$
    \item $\ldots\ldots + 7 = 14$
    \item $9 + 3 = \ldots\ldots$
    \item $-10 + \left( -10\right) = \ldots\ldots$
    \item $\ldots\ldots \times \left( -2\right) = -4$
    \item $-5 - \left( -2\right) = \ldots\ldots$
    \end{enumerate}
  \end{multicols}

    \section*{Équations}
  \textit{Résoudre les équations.} \textbf{Écrire les étapes.} Rédiger soigneusement.

  \begin{multicols}{2}
    \subsection*{ex2 - Équations 1}

    \begin{eqnarray*}
      & a) & x + 12 = 25  \\
      & b) & 3x + 10 = 40  \\
      & c) & 2x - 5  = 26  \\
      & d) & 8x - 12 = -45 \\
      & e) & -12x + 14 = 102 \\
      & f) & -15x - 16 = -24 \\
      & g) & -6x - 24 = 32  \\
      & h) & 10 + 5x = 22  \\
      & i) & 80x + 12 = 12 \\
      & j) & \sqrt{2} x + 12 = 40 
    \end{eqnarray*}


    \subsection*{ex3 - Équations 2}

    \begin{eqnarray*}
      & a) & 4x + 2 = 2x + 12    \\
      & b) & 8x + 12 = 6x - 4     \\
      & c) & 24x - 22 = 5x + 12    \\
      & d) & -12x + 13 = 4x -20     \\
      & e) & -8x - 10 = -12x - 24  \\
      & f) & 4,2x + 14 = 5x + 36    \\
      & g) & 5,4x + 21 = x - 14,5 \\
      & h) & 15 + 2x = 8x - 21,1  \\
      & i) & 9 - 2x = 104 +  x   \\
      & j) & \dfrac{5}{x} = \dfrac{10}{102}
    \end{eqnarray*}
  \end{multicols}

\end{document}
